\documentclass{article}
\begin{document}
\title{EFANNA : Extremely Fast Approximate Nearest Neighbor Search Algorithm - User manual}
\author{Cong Fu, 15267003518@163.com\\Deng Cai, dengcai@cad.zju.edu.cn}
\maketitle

\section{Introduction}
EFANNA is used for fast \textbf{Approximate Nearest Neighbor Search} (ANNS) problem based on a \textbf{Approximate $k$ Neatest Neighbor Graph} (approximate $k$NN graph).

ANNS problem can be defined as follows: Given a metric space $M$, a base point set $D=\{p_1, p_2, p_3, ...... , p_n\}$, a number of nearest neighbors of given new query point $q$ in $M$ should be returned quickly. However, searching for exact nearest neighbors with large scale and high dimensional data is too slow, and many proposed algorithms are even slower than brute-force search. Therefore, ANNS has drawn more and more attention. Given a number $k$, instead of finding the exact $k$ nearest neighbor of $q$, ANNS algorithms search for the most possible $k$ nearest neighbor for $q$. In other words, some points among the returned $k$ are not the true top $k$ neighbors. Due to sacrificing some accuracy, ANNS algorithms can achieve very high speed. EFANNA is much faster than all the previously proposed algorithms on ANNS.

EFANNA carries out ANNS based on a approximate $k$NN graph. A $k$ nearest neighbor graph is a big table, For each point $p$ in $D$, there is an entry recording $k$ nearest neighbor in $D$ in metric space $M$. However, it's also time costing to build an exact $k$NN graph. EFANNA can build approximate $k$NN graph quickly and outperforms previous algorithms, too.

So far EFANNA is written in C++. Interfaces in other languages like Python and MATLAB will be provided soon. EFANNA now supports SSE instructions acceleration. OpenMP will be supported soon for more speed-up.
\end{document}